%article=mwart;book=mwbk;report=mwrep
%\documentclass[10pt,oneside]{mwbk} %polish style
\documentclass[10pt]{article} %international style
\usepackage[utf8]{inputenc}

\usepackage[margin=11mm]{geometry}
\usepackage{color}
\usepackage{graphicx}
\graphicspath{ {./img/} }
\usepackage{float}

%%%---------------------------
% Rectangle border for scanner
%%%---------------------------
\usepackage{tikz}
\usetikzlibrary{calc}
\usepackage{eso-pic}
\AddToShipoutPictureBG{%
\begin{tikzpicture}[overlay,remember picture]
\draw[line width=8pt]
    ($ (current page.north west) + (1cm,-1cm) $)
    rectangle
    ($ (current page.south east) + (-1cm,1cm) $);
\end{tikzpicture}
}
%%%---------------------------


\usepackage{array}
\usepackage{wrapfig}
\usepackage{multirow}
\usepackage{tabularx}
\usepackage{xcolor}
\usepackage{nicematrix}
\usepackage{array} % and/or
\usepackage{longtable} % and/or
\usepackage{colortab} % or
\usepackage{colortbl}
\usepackage{arydshln} % below some settings

\setlength{\dashlinedash}{1pt} % dash length
%\setlength{\dashlinegap}{4.5pt}
\setlength{\dashlinegap}{2.5pt} % gap between dashes
\setlength{\arrayrulewidth}{0.7pt} % line thickness



\begin{document}
\pagenumbering{gobble} % no page number

%%%---------------------------
% Bottom page necessary images
%%%---------------------------
\begin{tikzpicture}[overlay, remember picture]
\node [shift={(92mm,-242mm)}]  %at (current page.south west)
{%
%\fbox{\includegraphics[width=190mm]{img/!bottom+qr.jpg}}};
\includegraphics[width=191mm]{img/!bottom+qr.jpg}
};
\end{tikzpicture}
%%%---------------------------

%%%---------------------------
% PUT HERE ANYTHING ELSE
%%%---------------------------

%%%---------------------------
% inserting simple table 2 cols
%%%---------------------------

\vspace{1cm}

\renewcommand{\arraystretch}{3.5} % vertical width of the row/cell (height of the row)
\begin{center}


 \begin{tabularx}{\textwidth} { X : X } % { column separator(: |) column }
\hdashline
  &   \\ % write as many separators (&) as above (:)
\hdashline
   &   \\
\hdashline
  &   \\
\hdashline
   &   \\
\hdashline
  &   \\
\hdashline
   &   \\
\hdashline
  &   \\
\hdashline
   &   \\
\hdashline
  &   \\
\hdashline
   &   \\
\hdashline
   &   \\
\hdashline
  &   \\
\hdashline
   &   \\
\hdashline
   &   \\
\hdashline

\end{tabularx}
\end{center}




\end{document} 